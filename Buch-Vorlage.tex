% LaTex Buch Vorlage von Inga Wolter & Fabian Schumann
% latexlovestory.de

\documentclass[a5paper 	%Papierformat
,DIV=13		%Einstellungen für den Satzspiegel
,twoside		%zweiseitiger Satzspiegel; bei der
						%book-Dokumentklasse automatisch
						%so eingestellt, dass neue Kapitel
						%immer auf der rechten (ungeraden)
						%Seite beginnen
,11pt % Schriftgröße
,headsepline	%erzeugt eine Trennlinie in der
						%Kopfzeile. Gibts natürlich auch
						%als "footsepline"
]{scrbook} 
\usepackage{csquotes} % für Anführungszeichen
  \usepackage{setspace}	
  \usepackage{microtype}


\ifdefined\HCode
  % EPUB-/HTML-Modus: keine Packages für Formatierung laden
\else
  % PDF-Modus: lade Packages für Formatierung & Co.
  \usepackage{fontspec} % Schriftarten
  \usepackage{ellipsis} % verbesserter Abstand bei 3 Punkten ...
  \usepackage[osf]{libertine} 	% deaktivieren, falls Schriftart
						% "Libertine" nicht installiert ist.
						% Die Option "osf" (old school font)
						% bewirkt, dass alle Ziffern als
						% Minuskelziffern (Mediävalziffern)
						% gesetzt werden.
\fi
\usepackage[ngerman]{babel} 
\usepackage{lettrine} 	%Initiale 
%\usepackage{xpatch} % for \xpatchcmd
%\usepackage{txfonts} % Herzsymbol
\usepackage[colorlinks=false,pdfborder={0 0 0},bookmarksnumbered]{hyperref}

\begin{document}

\hypersetup{
	pdftitle={Latex Love Story - Wie schreibe ich ein Buch mit \LaTeX{}},
	pdfauthor={von Inga Wolter \& Fabian Schumann},
	pdfsubject={}
}
	

\ifdefined\HCode
        % EPUB-/HTML-Modus
        \title{LaTeX Love Story}
        \author{Inga Wolter \& Fabian Schumann}
        \date{2025}
        \maketitle
\else
        \begin{titlepage}
                % PDF-Modus:
                \subject{}
                \title{\Huge \textsc{LaTeX Love Story}}
                \subtitle{Wie schreibe ich ein Buch mit \LaTeX{}}
                \author{von Inga Wolter \& Fabian Schumann}
                % Kontakt ergänzen
                % Website: \href{https://latexlovestory.de}{latexlovestory.de}
                \date{2025}
        \end{titlepage}
        \maketitle
\fi
\onehalfspacing % 1,5-facher Zeilenabstand

\frontmatter % Gliederungsoption bei Dokumentklasse Book für alles, das vor dem Haupttext kommt

\tableofcontents

\mainmatter % Gliederungsmöglichkeit bei Dokumentklasse Book für den Hauptteil

\chapter{Einleitung}

\section{ Was ist überhaupt dieses \LaTeX{}?}

Willkommen zu \textit{Latex Love Story}! Super, dass du dieses Buch in die virtuelle Hand nimmst! ;)
Das freut uns wirklich sehr, da wir viel Zeit und Mühe in die Erstellung gesteckt haben.
Es soll dir helfen, die Grundlagen von \LaTeX{} zu verstehen und dir den Einstieg in die Arbeit mit \LaTeX{} zu erleichtern.

\LaTeX{} ist ein mächtiges Werkzeug, das dir viele Möglichkeiten bietet, aber es kann auch komplex sein.
Es ist ein Textsatzsystem, das ursprünglich für wissenschaftliche Arbeiten entwickelt wurde.
Dabei ist es besonders gut geeignet für Dokumente, die viele mathematische Formeln, Tabellen und Abbildungen enthalten.
\LaTeX{} ist nicht nur ein Textverarbeitungsprogramm, sondern ein System, das dir hilft, Dokumente in hoher typografischer Qualität zu erstellen.


\section{Mindset}
\begin{quote}
    \textit{Es ist nicht wichtig, wie du anfängst, sondern wie du weitermachst.}
\end{quote}

Ärgere dich nicht, wenn du am Anfang nicht alles verstehst oder wenn etwas nicht funktioniert.
Das ist normal und gehört zum Lernprozess dazu.
Es wird vorkommen, dass die Generierung der PDF- oder EPUB-Datei mit einem Fehler abbricht.
Das ist frustrierend, aber es ist auch eine Gelegenheit, etwas Neues zu lernen.

Die Lernkurve bei \LaTeX{} kann anfangs steil erscheinen, aber mit etwas Übung wirst du schnell Fortschritte machen.
Und sei dir sicher - andere haben genau die gleichen Probleme wie du.
Deshalb nutze gern bei jeglicher Art von Fragen unseren \href{https://chatgpt.com/g/g-68386eb107ac8191aa5beeb7ce8f2fdc-latex-schreibassistent}{LaTeX{} Schreibassistenten}.

\section{Warum dieses Buch?}

Dieses Buch soll dir helfen, die Grundlagen von Latex zu verstehen und dir den Einstieg in die Arbeit mit \LaTeX{} zu erleichtern.
Es ist selbstverständlich komplett mit Latex geschrieben und erstellt worden.
Deshalb ist es gleichzeitig auch eine ideale Vorlage, die du für dein eigenes erstes Buchprojekt mit \LaTeX{} verwenden kannst.
Falls du damit direkt loslegen möchtest, kannst du dir das Vorgehen dazu gern im  \hyperlink{chapter.4}{Kapitel \textit{Erste Schritte}}  durchlesen.

\section{Arbeiten mit diesem Buch}

Dieses Buch ist als Einführung in die Arbeit mit \LaTeX{} gedacht.
Es soll dir helfen, die Grundlagen zu verstehen und erste Schritte in der Erstellung von Dokumenten mit \LaTeX{} zu machen.
Die Kapitel sind so strukturiert, dass du sie in der Reihenfolge lesen kannst, die für dich am sinnvollsten ist.
Die Beispiele sind so gewählt, dass sie leicht nachvollziehbar sind und dir helfen, die Konzepte zu verstehen.

\chapter{Installation}

\section{Installation unter Windows}
\subsection{Vorbereitung}
\begin{itemize}
    \item Stelle sicher, dass du über Administratorrechte verfügst.
    \item Schließe alle \TeX-bezogenen Programme, falls bereits eine ältere Version installiert ist.
\end{itemize}

\subsection{Herunterladen von MiK\TeX}
Das Installationspaket unter Windows heisst \enquote{MiK\TeX}. Es wird gern genutzt, da es einfach zu installieren und zu verwenden ist.
\begin{enumerate}
    \item Rufe \mbox{\url{https://miktex.org/download}} im Browser auf.
    \item Wähle den Installer für Windows und klicke auf \enquote{Download}.
    \item Speichere die Datei (z.\,B.\ \texttt{MiKTeX 24.1 x64.exe}) in einem Verzeichnis deiner Wahl.
\end{enumerate}

\subsection{Installation von MiK\TeX}
\begin{enumerate}
    \item Navigiere im Windows-Explorer zu dem Ordner, in dem du den Installer gespeichert hast.
    \item Doppelklicke auf den Dateinamen \texttt{MiKTeX 24.1 x64.exe}, um die Installation zu starten.
    \item Im Setup-Assistenten wählst du:
    \begin{itemize}
        \item \enquote{Install MiK\TeX}
        \item \enquote{I accept the MiK\TeX copying conditions} $\rightarrow$ \enquote{Next}
        \item \enquote{Install for: Only for me} oder \enquote{All users} je nach Bedarf.
        \item Lege das Installationsverzeichnis fest oder verwende den Standardpfad.
        \item Klicke auf \enquote{Next} und warte, bis die Installation abgeschlossen ist.
    \end{itemize}
    \item Nach Abschluss klickst du auf \enquote{Close}. MiK\TeX ist nun installiert.
\end{enumerate}

\subsection{Paketverwaltung und Update}
\begin{itemize}
    \item \enquote{MiK\TeX Console} öffnen (über das Startmenü).
    \item Unter \enquote{Updates} klickst du auf \enquote{Check for updates}, um verfügbare Updates für die vorinstallierten Pakete herunterzuladen und zu installieren.
    \item Stelle sicher, dass die Option \enquote{Always install missing packages on-the-fly} aktiviert ist.
\end{itemize}

\subsection{Test der Installation}
\begin{enumerate}
    \item Öffne die Eingabeaufforderung (cmd.exe).
    \item Gib \texttt{pdflatex --version} ein und bestätige mit Enter.
    \item Du solltest eine Ausgabe mit der Version von \TeX Live sehen, z.\,B. \enquote{MiK\TeX 24.1 (64-bit)}.
\end{enumerate}

\section{Installation unter macOS}
\subsection{Vorbereitung}
\begin{itemize}
    \item Viele macOS-Nutzer verwenden \emph{Mac\TeX}, eine speziell für macOS angepasste TeX Live-Distribution.
    \item Stelle sicher, dass du macOS Catalina (10.15) oder neuer verwendest.
    \item Schließe alle \TeX-bezogenen Programme, falls bereits eine ältere Version installiert ist.
\end{itemize}

\subsection{Herunterladen von Mac\TeX}
\begin{enumerate}
    \item Rufe \mbox{\url{https://tug.org/mactex/}} auf.
    \item Klicke auf \enquote{MacTeX Download} und wähle die aktuelle \texttt{.pkg}-Datei.
    \item Optional: Wenn du wenig Speicherplatz hast, lade \enquote{BasicTeX} 
    
    (\mbox{\url{https://tug.org/mactex/morepackages.html}}) herunter.
\end{enumerate}

\subsection{Installation von Mac\TeX}
\begin{enumerate}
    \item Öffne die heruntergeladene \texttt{MacTeX.pkg}-Datei per Doppelklick.
    \item Folge dem Installationsassistenten:
    \begin{itemize}
        \item \enquote{Fortfahren}, Lizenzbedingungen akzeptieren
        \item Zielvolume wählen, \enquote{Installieren}
        \item Passwort eingeben zur Bestätigung
    \end{itemize}
    \item Erfolgsnachricht: \enquote{The installation was successful.} → \enquote{Schließen}
\end{enumerate}

\subsection{Umgebungsvariablen}
\begin{itemize}
    \item Pfad: \texttt{/Library/TeX/texbin} (\texttt{echo \$PATH})
    \item Wenn nicht vorhanden, füge folgendes ein:
    \begin{verbatim}
export PATH="/Library/TeX/texbin:$PATH"
    \end{verbatim}
    \item Dann:
    \begin{verbatim}
source ~/.bash_profile   # oder ~/.zshrc
    \end{verbatim}
\end{itemize}

\subsection{Paketverwaltung und Update}
\begin{itemize}
    \item Terminal öffnen:
    \begin{verbatim}
sudo tlmgr update --self --all
    \end{verbatim}
    \item Weitere Pakete mit:
    \begin{verbatim}
sudo tlmgr install <Paketname>
    \end{verbatim}
\end{itemize}

\subsection{Test der Installation}
\begin{enumerate}
    \item Terminal öffnen:
    \begin{verbatim}
pdflatex --version
    \end{verbatim}
    \item Testdatei wie oben erstellen und kompilieren:
    \begin{verbatim}
pdflatex test.tex
    \end{verbatim}
\end{enumerate}

\section{Installation unter Linux}

Die Installation ist über einen Paketmanager oder direkt von TeX Live möglich. 
Im folgenden wird die Installation für Debian / Ubuntu beschrieben. 
Falls Unterstützung für ein anderes Derivat erforderlich ist, hilft mit Sicherheit unsere GPT gern weiter. % TODO Link einfügen

\subsubsection{Debian / Ubuntu}
\begin{enumerate}
    \item Terminal öffnen:
    \begin{verbatim}
sudo apt update
sudo apt install texlive
sudo apt install texlive-latex-extra texlive-fonts-recommended texlive-lang-german
sudo apt install texlive-full
    \end{verbatim}
\end{enumerate}

\subsubsection{Fedora / CentOS / RHEL}
\begin{verbatim}
sudo dnf install texlive-scheme-full
sudo dnf install texlive-scheme-basic
\end{verbatim}

\subsubsection{Arch Linux / Manjaro}
\begin{verbatim}
sudo pacman -S texlive-most
sudo pacman -S texlive-core
\end{verbatim}

\subsection{Direktinstaallation TeX Live}
\begin{itemize}
    \item Rufe \mbox{\url{https://www.tug.org/texlive/acquire-netinstall.html}} auf.
\end{itemize}
\begin{verbatim}
tar -xvzf install-tl-unx.tar.gz
cd install-tl-<Datum>
sudo ./install-tl
\end{verbatim}

\subsubsection{Eintragen in den PATH}
\begin{itemize}
    \item Pfad: \texttt{/usr/local/texlive/20XX/bin/x86\_64-linux}
    \item In \texttt{.bash\_profile} oder \texttt{.zshrc} einfügen:
    \begin{verbatim}
export PATH="/usr/local/texlive/20XX/bin/x86_64-linux:$PATH"
    \end{verbatim}
    \item Dann:
    \begin{verbatim}
source ~/.bash_profile   # oder ~/.zshrc
    \end{verbatim}
\end{itemize}

\subsection{Test der Installation}
\begin{enumerate}
    \item Terminal öffnen:
    \begin{verbatim}
pdflatex --version
    \end{verbatim}
    \item Testdatei wie zuvor kompilieren:
    \begin{verbatim}
pdflatex test.tex
    \end{verbatim}
\end{enumerate}

\section{Fehlerbehebung und Tipps}
\subsection{Pfadprobleme}
\begin{itemize}
    \item \texttt{which pdflatex} zur Pfadkontrolle verwenden
    \item Terminal nach PATH-Anpassung neu starten
\end{itemize}

\subsection{Speicherplatz}
\begin{itemize}
    \item Basisinstallation und dann Pakete bei Bedarf mit \texttt{tlmgr} oder MiK\TeX Console
\end{itemize}


%\backmatter

\end{document}%%