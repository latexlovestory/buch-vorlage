\chapter{Einleitung}

\section{ Was ist überhaupt dieses \LaTeX{}?}

Willkommen zu \textit{Latex Love Story}! Super, dass du dieses Buch in die virtuelle Hand nimmst! ;)
Das freut uns wirklich sehr, da wir viel Zeit und Mühe in die Erstellung gesteckt haben.
Es soll dir helfen, die Grundlagen von \LaTeX{} zu verstehen und dir den Einstieg in die Arbeit mit \LaTeX{} zu erleichtern.

\LaTeX{} ist ein mächtiges Werkzeug, das dir viele Möglichkeiten bietet, aber es kann auch komplex sein.
Es ist ein Textsatzsystem, das ursprünglich für wissenschaftliche Arbeiten entwickelt wurde.
Dabei ist es besonders gut geeignet für Dokumente, die viele mathematische Formeln, Tabellen und Abbildungen enthalten.
\LaTeX{} ist nicht nur ein Textverarbeitungsprogramm, sondern ein System, das dir hilft, Dokumente in hoher typografischer Qualität zu erstellen.


\section{Mindset}
\begin{quote}
    \textit{Es ist nicht wichtig, wie du anfängst, sondern wie du weitermachst.}
\end{quote}

Ärgere dich nicht, wenn du am Anfang nicht alles verstehst oder wenn etwas nicht funktioniert.
Das ist normal und gehört zum Lernprozess dazu.
Es wird vorkommen, dass die Generierung der PDF- oder EPUB-Datei mit einem Fehler abbricht.
Das ist frustrierend, aber es ist auch eine Gelegenheit, etwas Neues zu lernen.

Die Lernkurve bei \LaTeX{} kann anfangs steil erscheinen, aber mit etwas Übung wirst du schnell Fortschritte machen.
Und sei dir sicher - andere haben genau die gleichen Probleme wie du.
Deshalb nutze gern bei jeglicher Art von Fragen unseren \href{https://chatgpt.com/g/g-68386eb107ac8191aa5beeb7ce8f2fdc-latex-schreibassistent}{LaTeX{} Schreibassistenten}.

\section{Warum dieses Buch?}

Dieses Buch soll dir helfen, die Grundlagen von Latex zu verstehen und dir den Einstieg in die Arbeit mit \LaTeX{} zu erleichtern.
Es ist selbstverständlich komplett mit Latex geschrieben und erstellt worden.
Deshalb ist es gleichzeitig auch eine ideale Vorlage, die du für dein eigenes erstes Buchprojekt mit \LaTeX{} verwenden kannst.
Falls du damit direkt loslegen möchtest, kannst du dir das Vorgehen dazu gern im  \hyperlink{chapter.4}{Kapitel \textit{Erste Schritte}}  durchlesen.

\section{Arbeiten mit diesem Buch}

Dieses Buch ist als Einführung in die Arbeit mit \LaTeX{} gedacht.
Es soll dir helfen, die Grundlagen zu verstehen und erste Schritte in der Erstellung von Dokumenten mit \LaTeX{} zu machen.
Die Kapitel sind so strukturiert, dass du sie in der Reihenfolge lesen kannst, die für dich am sinnvollsten ist.
Die Beispiele sind so gewählt, dass sie leicht nachvollziehbar sind und dir helfen, die Konzepte zu verstehen.
